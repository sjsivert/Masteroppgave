
\section{Loss functions}

The loss function is critical in order to assess the error of a deep learning model.
Different loss functions, such as those defined in \ref{section:BT:Loss},
are commonly used as a preformance metric in artificial inteligence.
Dependent on the application of the machine learning problem, different loss functions suite different usecases.

In order to find a fitting loss function for our problem, we need to assess other loss funtions as well.
This section will focus on a few specialized loss functions, spesificaly designed to work well with extreme values, such as thos we might incounter in our problem.

\subsection{DILATE}

% Time and shape distortion
Both MAE and MSE are proven to be robust loss functions for regular regression problems,
but a paper published by \citeauthor{Guen2019} highligts some problems when the shape of the target function matters
such as in time series.
\autoref{fig:dilate} shows three different fits to the same target function, that all have
the same MSE error.

\begin{figure}[h!]
  \centering
  \begin{subfigure}[b]{0.3\textwidth}
    \centering
    \caption{Non informative prediction}
    \label{fig:dilate-non-informative}
    \includegraphics[width=\textwidth]{./figs/illustrations/dilate_ex1.png}
    \hfill
  \end{subfigure}

  \begin{subfigure}[b]{0.3\textwidth}
    \centering
    \caption{Correct shape, but with time delay}
    \label{fig:dilate-correct-shape}
    \includegraphics[width=\textwidth]{./figs/illustrations/dilate_ex2.png}
    \hfill
  \end{subfigure}
  \begin{subfigure}[b]{0.3\textwidth}
    \centering
    \caption{Correct time, but inaccurate shape}
    \label{fig:dilate-correct-time}
    \includegraphics[width=\textwidth]{./figs/illustrations/dilate_ex3.png}
    \hfill
  \end{subfigure}
  \caption{Three examples of different shapes but same MSE error \citep{Guen2019}}
  \label{fig:dilate}
\end{figure}

In order to battle this problem they introduce a new loss function they call 
\textbf{DILATE (Distrortion Loss including Shape and Time)} \cite{Guen2019}.
DILATE uses a Neural Network instead of a mathematical function. 
It aims at accurately predicting sudden chanes, and explicitly incorporates two terms
supporting precise shape and temporal change detection.

The paper concludes that DILATE is comparable to the standard MSE loss when evaluated on MSE,
and far better when evaluated on time and shape metrics.

The DILATE loss function does seem promising for our problem. Predicting anomalies and their behavior
is both a time-sensitive and shape-sensitive problem. However adding another neural network as a loss function
might complicate an already complicated model even further.
If we have enough data to train such a complicated model remains to be seen. 


% Extreme value loss functions
\subsection{EVL}
\citeauthor{Ding2019} wrote a paper in 2019 regarding modern deep learning methods and their
weak performance when applied to real world time series, because their inability to predict extreme values.

Their deduction of why modern methods are unsatisfactory is bevause of the quadratic loss in 
the MSE loss function.
They take inspiration from \textit{Extreme Value Theory}, and develops a new kind of loss function
which they call \textbf{Extreme Value Loss (EVL)}.
They employ a memory network in order to memorize extreme events in historical records.

\autoref{fig:evl} show two fitted time series from the papers results. They conclude
that the method is superior to state-of-the-art methods in extreme event detection, and 
in time series prediction.

This loss function seem to capture our need to predict anomalies.
It does however has a few drawbacks. Mainly it that is complicates our model, and it needs training data for it's memory network. 
Which in our case is currently missing. 

\todo{Skrive litt mer detaljert om dette?}

\begin{figure}[h!]
  \centering
  \begin{subfigure}[b]{0.5\textwidth}
    \centering
    \caption{Output from GRU}
    \label{fig:evl-example1}
    \includegraphics[width=\textwidth]{./figs/illustrations/evl_example1.png}
    \hfill
  \end{subfigure}

  \begin{subfigure}[b]{0.5\textwidth}
    \centering
    \caption{Output from EVL model}
    \label{fig:evl-example2}
    \includegraphics[width=\textwidth]{./figs/illustrations/evl_example2.png}
    \hfill
  \end{subfigure}
  \caption{Figures from \cite{Ding2019}}
  \label{fig:evl}
\end{figure}