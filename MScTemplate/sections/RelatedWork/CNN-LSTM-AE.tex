% https://www.mdpi.com/1424-8220/20/5/1399/htm

\section{CNN with LSTM-Autoencoder}

By attempting to improve opon the predicitve accuracy of electricity forecasting,
the paper \cite{Khan2020} proposes a new predictive framework.
This framework aims at achieving an improved forecasting by introducing a new forecasting method.
The afformentioned method is a hybrid framework based on convolutional neural network, LSTM and Autoencoder,
creating a CNN with an LSTM-Autoencoder.

The paper explores the predictive abilities of the propsed model on two datasets conserning the electricity usage of Residential and comercial buildings.
% -> TODO: Double check that the two datasets do this....
By assessing the proposed model against current state of the art methods using several different loss metrics,
there is a clear advantage in predictive capabilities using the propsed methhod.

Unlike the LSTM, the hybrid LSTM-AE is able to learn from temporal dependencies between sequences.
Coupled with the convolutional neural network to extract data featrues, the improvement in loss reduction is to be expected.

% TODO -> Why does we not use this method?

\todo[inline]{Do we need to mention this method? There is no good reason for not selecting this method that i can se. We need a reason if we are to write this. In addition, related work should be about what is related to our method. If we are not using this method, is it realy related?o}