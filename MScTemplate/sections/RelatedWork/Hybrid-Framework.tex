
\section{Hybrid frameworks for Time series forecasting}
\label{section:RelatedWork:Hybrid}

% CNN = Feature extraction
% LSTM = Sequence learning

% Introduce hybrid methods and why they are a thing.
% -> Write about the ARIMA and BP Hybrid model.
% Introduce the two new methods
% -> CNN-AE and LSTM
% -> CNN and LSTM-AE

Improvements in time series forecasting have emerged in the last few years with the introduction of deep learning.
Previous state-of-the-art methods, such as the ARIMA model, have been exchanged for deep learning methods using convolution and recurrent networks to improve accuracy.
One such improvement strategy has been to introduce hybrid models.
These models are comprised of different models with unique abilities, helping to increase the accuracy of the connected model.


One example of such a hybrid model is the ARIMA-BP model explored in \cite{Bowen2020}.
This paper explores the predictive abilities of the ARIMA model and a Back-Propagation Neural network on a time series problem of sales forecasting.
The ARIMA and BP models were applied to the problem, with the ARIMA somewhat outperforming the BP model.
However, by combining the two models, the hybrid model was able to achieve a lower predictive accuracy than any of the two models on their own.
This hybrid model is heavily based on the state-of-the-art statistical method of ARIMA.
As we have explored in \ref{section:RelatedWork:Statistical-NN}, the introduction of more advanced neural networks often achieves better predictive results than the old state-of-the-art ARIMA model.
We, therefore, argue that it should make sense to look further into other hybrid models, utilizing these new and improved models.
By combining methods such as Convolution and Recurrent networks in a hybrid model, it should yield a better result than they would on their own.
% ARIMA-BP er en hybrid model som gir bedre resultater enn de enktelte delene alene. Gir et godt grunnlag for at hybrid modeller er bra!


This combination of models is explored in a few different ways.
One such hybrid model is the "CNN and LSTM-Autoencoder" method explored in \cite{Khan2020}.
This method is used to explore the predictive ability of the hybrid model on electricity forecasting in residential and commercial buildings.
The findings of the paper conclude with the same result as we propose above.
The aforementioned method is a hybrid framework based on convolution, LSTM, and Autoencoder.
Convolution was used to extract features from the input data before the LSTM-Autoencoder attempt to extract temporal dependencies between the sequences.
The proposed method was applied to the electricity forecasting problem alongside other models such as ARMA, SVM, SVR, and others.
The framework outperformed the state-of-the-art models at prediction using several different performance metrics such as MSE, MAE, RMSE, and MAPE.


Similar to the framework described above, yet another method utilizing the same methods was defined in \cite{Zaho2019}.
% TODO! Add correct citation!!!!
This model differs in the selected architecture, creating a Convolutional Autoencoder instead of the previous LSTM Autoencoder.
The Convolutional Autoencoder is used to extract and reduce dimensional features before the LSTM extracts the temporal features.
Similar to the previously mentioned method, this "CNN-AE LSTM" hybrid method is tested against several other current predictive models.
The proposed method outperforms methods such as LSTM, ARIMA, and SVR, achieving a much lower predictive error than the other methods.


% TODO: -> Update the section reference to "Statistical vs NN"
\ref{section:RelatedWork:Statistical-NN} explores the predictive ability of statistical methods in comparison to deep learning methods.
This section discusses how deep learning methods such as LSTM and Convolution work well in order to predict in cases where there is enough data.
Considering our problem space and available data, this applies in this case.
Thus, the usage of deep learning methods such as LSTM and Convolution is to be preferred.
The proposed hybrid frameworks outperform deep learning methods such as LSTM.
With this in mind, it is clear that a hybrid method is well suited for predictions in our problem space.
A hybrid framework has already been shown to exceed the predictive capabilities of its individual parts.
This was done with the ARIMA-BP model explored in \cite{Bowen2020}, showing that the connected hybrid method performed better than the ARIMA or BP models could on their own.
It is likely that this would also be the case for a hybrid CNN and LSTM model.

The proposed problem-space results in data with high fluctuations and noise.
In order to increase the predictive abilities, a method well suited for working with data with high noise should be selected.
A CNN-AE model should be able to solve this problem.
The CNN is able to extract the spatial features of the data while the AE can filter out the noise and fluctuations in the data.
One such hybrid framework should therefore be well suited for the task at hand.
% It is therefor well suited...
\todo[inline]{Stemmer det at det er mye noise i dataen? Eller burde vi kalle det noe annet?}


%The proposed methods for hybrid networks using CNN, Autoencoder and LSTM, has shown to excide the predicite capabilities of methods such as ARIMA and LSTM.
