\subsection{Proposed method framework}

% Hva er vår propsed method framework?
% Her går vi ikke så mye inn på det med Global vs Local, DETTE GJØRES SENERE!
% Her blir det mer viktig å gå inn på typen data vi har, ikke relasjoner mellom den.
% Husk at vi ikke har beskrevet dataen vår, og vi kommer sansyneligvis ikke til å gjøre dette heller
% Vær derfor overordnet og ikke gå for mye i detalje, men vær litt overordnet beskrivende
% Det viktigste her er at vi får frem hvorfor det er denne hybrid-metoden som er brukt, og ikke noe annet
% Hva er grunnen til at denne methoden ble valgt over den andre CNN og LSTM-AE metoden som også er presentert i related work?
% Hva er grunnen til at en vanlig LSTM methode ikke er brukt her?
% Hva er grunnen til at vi bruker både encoder-decoder biten av en Autoencoder, og ikke bare encoder biten. (NB! Her er jeg litt usikker selv, men det kan forsvares!)
% Få frem hvorfor dette er en god ide, og hvordan det relaterer til andre løsninger!

With the aim of answering the goal presented in section \ref{section:Introduction:Goal} a model framework is proposed.
The proposed method is intended to achieve higher predictive ability than other state-of-the-art methods on the presented dataset.
The presented model is a Convolutional autoencoder with and LSTM.
With the ARIMA model and the LSTM model as benchmarks, the proposed model aims at achieving lower predictive error than these methods.
The attained dataset presented in section ... is a part of the e-commerce domain, containing highly fluctuating data, seasonality, and more.
Such a dataset presents difficulties in prediction with high accuracy, and specific models might therefore attain better results than standard state-of-the-art methods.

Section \ref{sections:RelatedWork:Hybrid} discusses the improvements in predictive accuracy achivable with the use of hybrid models.
Utilizing methods for extracting borth temproal and spacial values apeared to yeald improvements.
With the available dataset containing large fluctuations in values, the designed framework would need to be able to achieve predictions despite this.
Therefor the proposed model is heavily influenced by the time series prediction model presented in \cite{Zhao2019}.
The model is intended to work with highly fluctuating datasets, as the paper supplies results showing a decrease in predictive error with the hybrid method over a LSTM model.
Similarly, the available dataset described in section ... share these atributes.
Therefor, the assertion is made that the same concepts applied in \cite{Zhao2019} can be applied in this problem space.
Such a hybrid model would then be introduced to a new problemspce, the e-commerce sector, in order to make predictions.
As far as we can tell, such a model has not previously been applied with e-commerce prediction.

% Why are we not using a regular LSTM method?
Section ... referes to the previously applied LSTM networks used in e-commerce time series prediciton.
Despite the ARIMA beeing commonly used in time series prediction, the LSTM network has shown great predictive accuracy.
The aim of the method arcitecture therefor to explore an improved LSTM model in order to achieve better results.
Section ... discusses how complex hybrid models are able to achieve higher accuracy than their individual parts.
One such example is the use of LSTM in \cite{Zhao2019}, achieving higher prediction error than the propsed hybrid model, using multiple loss metrics.
In order to increase predictive accuracy, and reduce error, such hybrid methods is thereofr worth considering.


% Why are we using this CNN-AE + LSTM, insted of another method?
There are multiple different types of hybrid methods, all with different specialisations and usecases.
Du to the data distribution of the dataset, the method presented in \cite{Zhao2019} was selected as the basis for the model.
Although the proposed CNN-AE and LSTM model has already been applied to other datasets, the aim of this paper is to evaluate the use of it on a new e-commerce problem space.
The the following sections, other aspects of the framework will be introduced.
This includes the model structure selected, discussing the implications of a global method, or a multivariate approach.
These alterations from the model proposed in \cite{Zhao2019} is intended to improve the applicability of the model on this e-commerce problemspace.

