\section{Data Exploration}
\label{section:Architecture:DataExploration}
In this section we will describe the main dataset given to us 
by Prisguiden.no. The full Jupyter notebook exploration
is located at [TODO Add source]. This section will only
describe the highlights from that Notebook.

As of \textit{23-11-2021} the dataset consists of 35966868 entries.
The oldest recorded data are from the beginning of 2019.
It consists of 9 features.
as shown in \autoref{table:features_market_insight_dataset}.
\begin{table}[htbp]
  \centering
  \caption{Features of the Market insight dataset}
  \label{table:features_market_insight_dataset}
  \begin{tabular}{|c|l|c|l|}\hline\hline
   Nr & Name & Type & Description \\ \hline 
   1 &id &(int64) & Unique identifier \\ \hline
  2 & product id & (int64) & Associated Product id \\ \hline
  3 & manufacturer id & (int64) & Manufacturer id \\ \hline
  4 &cat id & (int64) & Associated category id \\ \hline
  5 & root cat id & (int64) & Associated root category id \\ \hline
  6 & date & (datetime64[ns]) & The date the data was captured \\ \hline
  7 & hits & (int64) & How many times the product page was visited \\ \hline
  8 &clicks & (int64) & How many times users clicked to a retailer \\ \hline
  9 & last modified & (object) & When the row was last modified \\ \hline
  \end{tabular}
\end{table}

The dataset consists of 1325 unique categories and 310499 unique products.
Each product is associated with a product category, for example
all CPUs are associated with the category
\textit{Prosessor (CPU)}.
Each product category is a leaf node of a category hierarchy.
The category \textit{Prosessor (CPU)} is a child of the parent category 
\textit{Datakomponenter}, which itself is a child of \textit{Data}.

Prisguiden.no are not interested in insight on the product level, but on the product category level.
Summing togheter all product clicks and hits to its closest parent category
gives us a table on the format shown in \autoref{table:market_insights_overview_11-12-21}.

\import{./tables/code_generated/data_exploration/}{market_insights_overview_11-12-21.tex}

Plotting some 
%\begin{figure}[h!]
%    \centering
%    \includegraphics[width=\textwidth]{./figs/code_generated/data_exploration/lineplot_1135    MediaspillerName: title, dtype: object.png}
%    \hfill
%    \caption{}
%    \label{fig:}
%\end{figure}

