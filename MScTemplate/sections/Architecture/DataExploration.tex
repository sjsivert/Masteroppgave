\section{Data Exploration}
\label{section:Architecture:DataExploration}
In this section we will describe the main dataset given to us 
by Prisguiden.no. The full Jupyter notebook exploration
is located at [TODO Add source]. This section will only
describe the highlights from that Notebook.

\subsection{Basic data structure}
As of \textit{23-11-2021} the dataset consists of 35966868 entries.
The oldest recorded data are from the beginning of 2019.
It consists of 9 features.
as shown in \autoref{table:features_market_insight_dataset}.
\begin{table}[htbp]
  \centering
  \caption{Features of the Market insight dataset}
  \label{table:features_market_insight_dataset}
  \begin{tabular}{|c|l|c|l|}\hline\hline
   Nr & Name & Type & Description \\ \hline 
   1 &id &(int64) & Unique identifier \\ \hline
  2 & product id & (int64) & Associated Product id \\ \hline
  3 & manufacturer id & (int64) & Manufacturer id \\ \hline
  4 &cat id & (int64) & Associated category id \\ \hline
  5 & root cat id & (int64) & Associated root category id \\ \hline
  6 & date & (datetime64[ns]) & The date the data was captured \\ \hline
  7 & hits & (int64) & How many times the product page was visited \\ \hline
  8 &clicks & (int64) & How many times users clicked to a retailer \\ \hline
  9 & last modified & (object) & When the row was last modified \\ \hline
  \end{tabular}
\end{table}

The dataset consists of 1325 unique categories and 310499 unique products.
Each product is associated with a product category, for example
all CPUs are associated with the category
\textit{Prosessor (CPU)}.
Each product category is a leaf node of a category hierarchy.
The category \textit{Prosessor (CPU)} is a child of the parent category 
\textit{Datakomponenter}, which itself is a child of \textit{Data}.

Prisguiden.no are not interested in insight on the product level, but on the product category level.
Summing togheter all product clicks and hits to its closest parent category
gives us a table on the format shown in \autoref{table:market_insights_overview_11-12-21}.

\todo{Update table when we get new data from Prisguiden}
\import{./tables/code_generated/data_exploration/}{market_insights_overview_11-12-21.tex}

\subsection{Category plot analysis}
\autoref{fig:lineplot1} and \autoref{fig:lineplot2} show hits and clicks
from a random sample of categories from 2019 to 2021.
Most of the plots show a clear yearly periodic pattern, as shown in 
\autoref{fig:lineplot-Hodetelefoner},
\autoref{fig:lineplot-Mobiltelefon},
\autoref{fig:lineplot-Julekalender}.
The scales between the categories are different,
\textit{Varmepumpe} \autoref{fig:lineplot-Varmepumpe} hower around 1000 hits a day,
while \textit{Mobiltelefon} \autoref{fig:lineplot-Mobiltelefon} gets around 7500 hits a day.
The scale within each category can differ vastly.
Spillkonsoller \autoref{fig:lineplot-Spillkonsoller} has a hit peak around 60000,
while it usually gets around 5000 hits a day.

Some categories, like \textit{Grafikkort GPU} \autoref{fig:lineplot-GPU}
breaks its relativly stable characterstics around christmas 2020,
and completely changes its behavior from that day forth.

\autoref{table:Mobiltelefon_statistics} shows some basic statistics of \textit{Mobiltelefoner}.
It has a mean of 6453 hits and a standard deviation of 1881 which is
around 30\% of the mean. This is quite a big variance in the dataset.

% Mobiltelefon stats table
\import{./tables/code_generated/data_exploration/}{Mobiltelefon_statistics.tex}


\begin{figure}[h!]
  \centering
  \caption{Category plots of hits and click rate from 2019-2021}
  \label{fig:lineplot1}
  \begin{subfigure}[b]{\textwidth}
      \includegraphics[width=\textwidth]{./figs/code_generated/data_exploration/lineplot_51_Hodetelefoner og ørepropper.png}
      \hfill
      \caption{Hits and clicks rate for \textit{Hodetelefoner og ørepropper}}
      \label{fig:lineplot-Hodetelefoner}
  \end{subfigure}

  \begin{subfigure}[b]{\textwidth}
      \includegraphics[width=\textwidth]{./figs/code_generated/data_exploration/lineplot_19_Mobiltelefon.png}
      \hfill
      \caption{Hits and clicks rate for \textit{Mobiltelefon}}
      \label{fig:lineplot-Mobiltelefon}
  \end{subfigure}

  \begin{subfigure}[b]{\textwidth}
      \includegraphics[width=\textwidth]{./figs/code_generated/data_exploration/lineplot_11781_Julekalender og adventskalender.png}
      \hfill
      \caption{Hits and clicks rate for \textit{Julekalender}}
      \label{fig:lineplot-Julekalender}
  \end{subfigure}
\end{figure}

\begin{figure}[h!]
  \centering
  \caption{Category plots of hits and click rate from 2019-2021}
  \label{fig:lineplot2}

  \begin{subfigure}[b]{\textwidth}
      \includegraphics[width=\textwidth]{./figs/code_generated/data_exploration/lineplot_43_Spillkonsoller.png}
      \hfill
      \caption{Hits and clicks rate for \textit{Spillkonsoller}}
      \label{fig:lineplot-Spillkonsoller}
  \end{subfigure}
  \begin{subfigure}[b]{\textwidth}
      \includegraphics[width=\textwidth]{./figs/code_generated/data_exploration/lineplot_30_gpu.png}
      \hfill
      \caption{Hits and clicks rate for \textit{Grafikkort GPU}}
      \label{fig:lineplot-GPU}
  \end{subfigure}
  
  \begin{subfigure}[b]{\textwidth}
      \includegraphics[width=\textwidth]{./figs/code_generated/data_exploration/lineplot_11054_Varmepumpe.png}
      \hfill
      \caption{Hits and clicks rate for \textit{Varmepumpe}}
      \label{fig:lineplot-Varmepumpe}
  \end{subfigure}
\end{figure}

\subsection{Correlation among categories}
A basic correlation analysis of all categories that has at least 100 data points to compare with each other,
gives us a correlation matrix, as shown in \autoref{fig:category_corelation_matrix}.
The ligher areas indicate a correlation closer to 1.0.
Darker areas indicate correlation closer to -1.0.

Based on the color spectrum from the matrix it is clear that categories cover almost the whole spectrum of correlation relationships.
The overall bright colors indicate a bias towards positive correlation.

  \begin{figure}[h!]
      \centering
      \includegraphics[width=\textwidth]{./figs/code_generated/data_exploration/category_correlation_matrix.png}
      \hfill
      \caption{Category correlation matrix}
      \label{fig:category_corelation_matrix}
  \end{figure}





