\subsection{ARIMA}
\label{section:Restuls:ARIMA}


\subsubsection{Tuning}
\label{section:Results:ARIMA:Tuning}
Tuning of the ARIMA models are done through the use of excessive grid search of parameters.
ARIMA parameters are found through the search of parameters for an interval for each of the p,d and q values.

\begin{table}[h]
  \centering
  \caption{The interval of values for the ARIMA parmaeters p,d,q used in grid search tuning.}
  \label{table:results:arima:tuning_parameter_interval}
  \begin{tabular}{|c|l|l|}\hline
    Parameter name & Interval start     & Interval end   \\ \hline
    p   & 1         & 8                 \\ \hline
    d   & 1         & 10                \\ \hline
    q   & 1         & 16                \\ \hline
  \end{tabular}
\end{table}

Tuning is conducted on all of the datasets defined in \cref{section:Architecture:Dataset}.
The tuning is done by calculating the error metric of the one-step ahed predictions done by the ARIMA model on the test set.
Multiple error metrics are used, calculating the MASE, SMAPE, MAE, and MSE of the predictions.
Tabel ... shows a small excerpt of the tuning metrics from dataset 1.
% TODO: Add an exerpt of the parameter tuning of dataset 1
% TODO: Add reference to complete tuning metrics in the apendix


\subsubsection{Experiments}

After completion of tuning as described in \cref{section:Results:ARIMA:Tuning},
the parameters yelding the best results are extracted and applied to experiments done on the dataset.

% TODO: Table of dataset and error metrics MAPE and SMAPE
% TODO: Add images of the dataset predictions
