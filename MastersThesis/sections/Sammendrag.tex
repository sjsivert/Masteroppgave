\section*{Sammendrag}
\label{section:Sammendrag}


Denne masteroppgaven utforsker bruk av en convolutions-autoencoder og LSTM modell for å gjennomføre
tidsserie prediksjon av produkt kategori trend data fra ``Prisguiden.no''.
Denne masteroppgaven arbeider mot å utvide den teoretiske kunnskapen om bruk av denne CNN-AE-LSTM modellen
ved å lage modeller som er både lokale og globale, samt ved bruk av en univariabel og multivariabel modell.
Resultatene fra disse eksperimentene er sammenlignet med resultater fra LSTM modeller av samme type, globale og lokale, univariable og multivariabel modeller.
I tillegg er disse modellene sammenlignet med en statistisk "baseline" ved bruk av den statistiske modellen SARIMA.


Resultatene fra disse eksperimentene viser at bruk av en lokal multivariabel LSTM modell er det som er best egnet for å gjennomføre
prediksjoner på dataen fra ``Prisguiden.no''.
Eksperimentene indikerer at CNN-AE-LSTM modellene er sterkt avhengig av type data som som skal predikeres,
og er spesielt egnet til bruk på data med store mengder støy.
Ved bruk av et datasett med mye støy indikerer eksperiment resultatene at CNN-AE-LSTM modellene utkonkurrerer LSTM modellen.
CNN-AE-LSTM modellen gjør det hakket bedre på data med mye støy, men er svært mye dårligere enn LSTM modellen på datasett med lite eller ingen datastøy.


CNN-AE-LSTM modellen er ikke velegnet for bruk til tidsserie prediksjoner på data fra ``Prisguiden.no''.
En lokal multivariabel LSTM modell er derimot bedre egnet for slike prediksjoner.

