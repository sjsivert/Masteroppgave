\subsection{Student's t-test}
\label{section:BT:t-statistic-test}
The \textit{student's t-test} is a statistical significance test.
It is a method of testing hypotheses about the mean of a small sample drawn from
a normally distributed population when the population standard deviation is unknown.
In other words, it is a method to check if the difference in mean between two groups,
is because of chance.

The method formulates a null hypothesis, which states that there is no effective difference
between the observed sample mean and the hypothesized population mean. In other words, it assumes that
both groups are equal.

It uses \Cref*{eq:t-formula-simplified} to calculate the t-value.
% \begin{equation}
%   \label{eq:t-formula}
%   t = \frac{\bar{x} - \mu}{s / \sqrt{n}}
% \end{equation}
% where $\bar{x}$ is the mean for one of the groups, $s$ is the standard deviation, $\mu$ is the mean from the other group with an
% unknown standard deviation,
% and $n$ is the sample size.

% A simplified version of the formula is presented in \Cref{eq:t-formula-simplified}.

\begin{equation}
  \label{eq:t-formula-simplified}
  t = \frac{\sum{D} / N} {\sqrt{\frac{\sum{D^2} - \frac{\sum{D}^2}{N}}{(N-1)(N)} } }
\end{equation}
where $D$ is the difference between each sample in the two groups , aka $x_i - y_i$.
So $\sum{D} = \sum{x_i - y_i}$

We can find the p-value by looking it up in the t-table by using our degrees of freedom, which are $N-1$.
If the p-value is below our $\alpha$ of 0.05 then we can reject the null hypothesis that the groups are equal.


