\subsection{Experiment 2 - LSTM Baseline}
\label{section:Method:LSTM}
The second experiment is meant to have a second baseline to compare the more
advanced neural network models against. The model is described in
\cref{section:BT:LSTM}. The experiment will try to answer
RQ[Insert research question here] \cref{TODO}.


\textbf{TODO: Insert RQ here.}

% What are we doing?
\subsubsection{Outline}
For each chosen category in [TODO]\cref{TODO} we are creating a Local Univariate LSTM(\textbf{LU-LSTM})
model. Each model will forecast a multistep ahead prediction of 1, 3, and 7 days ahead.
We will evaluate each model with the metrics MASE and SMAPE.

% How are we conducting the experient?
\subsubsection{Experiment details}
TODO

\textbf{Data preprocessing}: Input of the LSTM is only the "interest" variable.
This data is scaled down to a range between -1 and 1.

% What do we expect to happen?
\subsubsection{Expectations}
Each dataset contains at least 800 points of data.
This should be enough for a simple LSTM model to learn the underlying
behavior without overfitting. We expect the LU-LSTM method to outperform the LUA method.
Since this implementation only looks at each time series in isolation, we do
not think the results will be groundbreaking.

We expect the model to perform significantly worse on 3 day multi-step-ahead,
and be a lot worse than the naive forecast on 7 day multi-step-ahead.


%The ARIMA model is the first model implemented as a base-line model.
%As described in \Cref{section:BT:ARIMA}, the ARIMA model has long been a state of the art model for time sereies prediction.
%Due to this, the use of the ARIMA model as a base line, newer models can be compared to the previous state of the art model within the same problem space,
%and with the same dataset.
%With this, the first baseline model for this project is defined.

% TODO: Add references to the Codebase

%In order to achieve a baseline prediction on the selected dataset, the ARIMA model is tuned in order to achieve better predictive results.



\subsection{Results}
% TODO: Should be moved to the 'Results' chapter
The ARIMA model is created as a baseline model.
This is done in order to create a baseline prediction on the available datasets, using a well known method for time series forecasting.

Using the exchausive grid search presented in \Cref{section:Method:Arima:Tuning}, followed by manual tuning,
a set of parameters were found for the different time series in our dataset.

% TODO: Present results






