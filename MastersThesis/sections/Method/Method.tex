\chapter{Method}
\label{section:Method}

The applied methodology used in this thesis is explored in this chapter.
\Cref{section:Method:Hardware&Software} start with presenting the hardware used for running the experiments.
\Cref{section:Method:Metrics} then presents the error metrics used to measure the accuracy of the predictions across models.
Running experiments are done through the use of the framework developed. This framework is explored in more detail in \Cref{section:method:experiment-framework}.
% This is extended through \Cref{section:method:pytorch-vs-keras} exploring the selection of the machine learning framework Kerase and Tensorflow for this project.

Data processing steps used to manipulate data before the use with models are presented in \Cref{section:Method:Preprocessing}.

Model methods are then defined and explored. The SARIMA model is described in \Cref{section:Method:SARIMA}, the LSTM model in \Cref{section:Method:LSTM},
and the hybrid model CNN-AE and LSTM in \Cref{section:Method:CNN-AE-LSTM}.
Lastly, \Cref{section:Method:Statistical-t-test} discusses the use of the t-test for testing the statistical significance of results.


\import{./sections/Method/}{Hardware&Software.tex}
\import{./sections/Method/}{Metrics.tex}
\import{./sections/Method/}{Experiment-framework.tex}
%\import{./sections/Method/}{pytorch-vs-keras.tex}

\import{./sections/Method/}{DataProcessing.tex}

\import{./sections/Method/}{SARIMA.tex}
\import{./sections/Method/}{LSTM-method.tex}
\import{./sections/Method/}{CNNAE-LSTM.tex}
\import{./sections/Method/}{statistical-t-tests.tex}
