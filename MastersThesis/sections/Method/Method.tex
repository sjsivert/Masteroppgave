\chapter{Method}
\label{section:Method}

With the aim of implementation and evaluation of a new predictive method,

This chapter introduces the implemented method and methodology used in the practical research of time series prediction.
Initially, section \Cref{section:Method:Arima} and \Cref{section:Method:LSTM} presents the use and tuning of baseline methods needed in order to validate our method.
\dots

\section{Software and Hardware}
We used Python [TODO source] for the implementation of all the experiments.
The deep learning models are implemented using Keras [TODO source], because it seemed to perform better than PyTorch
  [\Cref{section:method:pytorch-vs-keras}].
We used Pandas [TODO Source], and NumPy [TODO source] for data manipulation.
We used genpipes [TODO source] to build data processing pipelines.
We used [TODO Sander, hva brukte vi for Arima?] library for the Arima model.

For hardware we used AMD Ryzen 5 5600X prosessor and 32 GB 32000 MHz memory.

\import{./sections/Method/}{Experiment-framework.tex}

\import{./sections/Method/}{DataProcessing.tex}

\import{./sections/Method/}{LSTM-method.tex}
% \import{./sections/Method/}{issues-with-lstm.tex}
\import{./sections/Method/}{pytorch-vs-keras.tex}


\import{./sections/Method/}{ARIMA.tex}
\import{./sections/Method/}{LSTM.tex}
\import{./sections/Method/}{AE.tex}
\import{./sections/Method/}{CNNAE-LSTM.tex}
