\section{Arima baseline}
\label{section:Method:Arima}

The ARIMA model is the first model implemented as a base-line model.
As described in \Cref{section:BT:ARIMA}, the ARIMA model has long been a state of the art model for time sereies prediction.
Due to this, the use of the ARIMA model as a base line, newer models can be compared to the previous state of the art model within the same problem space,
and with the same dataset.
With this, the first baseline model for this project is defined.

% TODO: Add references to the Codebase

In order to achieve a baseline prediction on the selected dataset, the ARIMA model is tuned in order to achieve better predictive results.

\subsection{Parameter tuning}
\label{section:Method:Arima:Tuning}
In order to find a fitting ARIMA model for the available dataset, tuning of parameters are important.
There are several methods available for model tuning, but only a few were selected for this project.

The first method of tuning were manual tuning of parameters.
Comprised of guess work and domain konwledge, a smal sett of model parameters were tested in order to find a fitting set of parameters.
This method is time consuming, but was usefull in cases where time constraints limited the use of excessive search of parameter sets.

% TODO: The use of Excessive Grid Search withing (1,7) (1,7) (1,11)
The secound method used was excessive grid search, testing each and every set of parameters within a range of parameter values.
With the ARIMA model, three parameters are configurable. p, d and q.
The grid search was run with parameters ranging from 1 to 7 for p and q, as well as 1 to 11 for q.
This parameter range is only limited by the computational power available to the researcher, as well as the limitation of time.


\subsection{Results}
% TODO: Should be moved to the 'Results' chapter
The ARIMA model is created as a baseline model.
This is done in order to create a baseline prediction on the available datasets, using a well known method for time series forecasting.

Using the exchausive grid search presented in \Cref{section:Method:Arima:Tuning}, followed by manual tuning,
a set of parameters were found for the different time series in our dataset.

% TODO: Present results




