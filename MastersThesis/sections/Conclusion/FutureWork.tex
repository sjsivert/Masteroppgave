
\section{Future Work}
\label{sections:Discussion:FutureWork}


This paper aimed to investigate the current landscape and solutions for time-series predictions as a whole,
as well as in an e-commerce setting.
Initially, the current predictive ability of solutions in an e-commerce setting was investigated.
It is then compared to the current state-of-the-art methods for prediction on time-series data,
as well as current methods for grouping and analyzing time-series data.
The paper focuses on a theoretical approach to solving the proposed research questions,
excluding practical experiments.

Work remains on conducting practical experiments on the time-series data supplied by Prisguiden,
in order to make time-series predictions on user trends.
This includes, but is not limited to, creating a predictive baseline using statistical models, such as the ARIMA method, a deep learning baseline such as an LSTM,
as well as implementing the framework proposed in \Cref{section:Architecture}.
In addition, practical experiments on the use of a local or global method with the available data,
as well as the use of a univariate or multivariate method.



