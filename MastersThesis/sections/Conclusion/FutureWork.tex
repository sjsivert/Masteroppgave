
\section{Future Work}
\label{sections:Discussion:FutureWork}

%%%%%%%%%%%%%%%%%%%%%%%%%%%%%%%%%%%%%%%%%%%%%%%%%%%%
% New Future work
% - Implement in real world application
% - Prisuiden can apply the resarch to data analysis
% - Further attempts done on splitting / decomposing data
% - Decomposing multivariate ...
% - Apply to a higher forecasting window (Predicting months insted of weeks)
% - There migh be changes in optimal approach hear.
% - We only breefly tried a 30 day forecast window
%%%%%%%%%%%%%%%%%%%%%%%%%%%%%%%%%%%%%%%%%%%%%%%%%%%%

% Implement real world
The work done in this paper investigates the use of the LSTM and convolutional autoencoder and LSTM
on the e-commerce domain of ``Prisguiden.no''.
However, the research done in this paper is not entirely practicaly applicable.
This comes from the fact that only a smal sub-set of the entirety of the available data has been applied
in the research done.
This was a limitation applied due to the scope of the paper,
as well as time limitaitons that addhear to the research done in a masters thesis.
Therefor, applying the models explored in this paper to a real world application still remains a valid future task.

% There exists other frameworks and models. Test others?
Additionaly, while the priority of this paper was to validate and explore the use of a convolutional autoencoder and lstm
in a time-series prediction setting, other models might also fit well with the task of making such predictions.
The Facebook model Prometheus, [TODO: Add more frameworks that can be applied \dots ],
are all valid models that can also be tested on the dataset to evaluate their usbility.
\todo[inline]{Add frameworks that can be used in addition to out methods. Facebook Prometheus is one. What are others?}


% Decomposition
As is described in \cref{section:Discussion}, different forms of data preprocessing were applied and tested in this paper.
While some types of decompositon was applied and tested, others were not tested due to limitations of time.
\textbf{STR decomposition} was only attempted on univariate models, while there multivariate models were not tested.
As the decomposition of univariate models improved the accuracy of the univariate preditions,
so might the use of decomposition with multivariate models, or other combinations and applications of decomposing the data.
\todo[inline]{Chech with what Sindre has written regarding decomposition \dots}


% Increase forecasting window
The research conducted in this paper focus primarily on the forecasting of a periode of 7 days.
While a 7 day prediciton does have value,
there might be some merits to increasing the forecasting window.
By increasing the forecasting window from 1 week to 30 days (a month), or 60 days (2 months),
the usecase for these predicitons would empower ``Prisguiden.no'' as discussed \cref{section:Discussion}.
While a 30 day prediction periode were breefly tested,
the 7 day prediction have been prioritized in this research.
Increasing the forecasting window is therefor a possible point of entry for furte resarch conducted on the
dataset from ``Prisguiden.no''.



%%%%%%%%%%%%%%%%%%%%%%%%%%%%%%%%%%%%%%%%%%%%%%%%%%%%
%%%%%%%%%%%%%%%%%%%%%%% OLD %%%%%%%%%%%%%%%%%%%%%%%%
%%%%%%%%%%%%%%%%%%%%%%%%%%%%%%%%%%%%%%%%%%%%%%%%%%%%

\iffalse
This paper aimed to investigate the current landscape and solutions for time-series predictions as a whole,
as well as in an e-commerce setting.
Initially, the current predictive ability of solutions in an e-commerce setting was investigated.
It is then compared to the current state-of-the-art methods for prediction on time-series data,
as well as current methods for grouping and analyzing time-series data.
The paper focuses on a theoretical approach to solving the proposed research questions,
excluding practical experiments.

Work remains on conducting practical experiments on the time-series data supplied by Prisguiden,
in order to make time-series predictions on user trends.
This includes, but is not limited to, creating a predictive baseline using statistical models, such as the ARIMA method, a deep learning baseline such as an LSTM,
as well as implementing the framework proposed in \Cref{section:Architecture}.
In addition, practical experiments on the use of a local or global method with the available data,
as well as the use of a univariate or multivariate method.
\fi
