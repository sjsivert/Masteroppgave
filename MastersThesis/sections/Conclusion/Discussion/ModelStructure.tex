\subsection{Model structure}

From what we learned from \Cref{section:RelatedWork:Model-structure}
there are many possible ways of constructing a model.
The four main categories we have discussed are: many local univariate models \cref{fig:model-example-local-univariate},
a global univariate model \cref{fig:model-example-global-univariate},
a local multivariate model \cref{fig:model-example-local-multivariate}, and a global univariate model.
\Cref{fig:model-structure-examples} shows three example-models for forecasting a set of $n$ time-series.
In the figure, each arrow indicates a time-series.

A local univariate approach would be similar to how an ARIMA model would work.
It would benefit from being simple, but to forecast $n$ product categories,
one would need $n$ models as seen in \cref{fig:model-example-local-univariate}.
Since consumer product categories are a dynamic set that is constantly changing, the benefit of having
$n$ models is that each change in a category will only affect its corresponding model.
If the creation of such a model requires little human intervention, this framework is scalable.
When a new category of products is created, a forecasting model is created with it. Its forecasting ability
would be limited but would gradually improve as more data accumulated.

\begin{figure}[h!]
  \centering
  \caption{Three model-structure examples. Each arrow is representing a time-series.}
  \label{fig:model-structure-examples}
  \begin{subfigure}[b]{0.4\textwidth}
    \includegraphics[width=\textwidth]{./figs/illustrations/illustration_local_univariate.png}
    \hfill
    \caption{Local univariate}
    \label{fig:model-example-local-univariate}
  \end{subfigure}
  \begin{subfigure}[b]{0.4\textwidth}
    \includegraphics[width=\textwidth]{./figs/illustrations/illustration_local_multivariate.png}
    \hfill
    \caption{Local multivariate}
    \label{fig:model-example-local-multivariate}
  \end{subfigure}
  \begin{subfigure}[b]{0.7\textwidth}
    \includegraphics[width=\textwidth]{./figs/illustrations/illustration_global_univariate.png}
    \hfill
    \caption{Global univariate}
    \label{fig:model-example-global-univariate}
  \end{subfigure}
\end{figure}

The greatest drawback of a local univariate approach would be the assumption that all product categories
are independent. We can expect some groups to correlate with each other. These interdependent relationships
could significantly improve forecasting ability.

A multivariate approach will capture these relationships.
However, one needs a thorough understanding of the underlying data to build meaningful models.
A thorough covariance analysis to identify correlating groups of product categories would be best.
%However, this might be outside of scope(?).
The work proposed by \cite{Sen2019} has the advantage of capturing interdependent time-series features and dealing with
some of the drawbacks of a multivariate model. However, their model is intended for as many time-series as up to millions.
This is to vastly over-engineer our problem.
%They solution is also based on a temoral convolution network.

A multivariate model does have some drawbacks in an ever-changing product category domain.
First, it's the issue that not all categories have the same life of existence, and thus a variable
amount of history data would amount to many null values.
Second, each time a category is added or changed, a covariance analysis has to be done to identify
which covariance group the category belongs to. Then that model has to be retrained.
% Depending on the number of changes to the category tree, this might hinder.
\Cref{fig:model-example-local-multivariate} shows an example model where the time-series set is split into two arbitrary buckets,
and one multivariate model is made for each bucket.

The univariate global approach, which is shown in \Cref{fig:model-example-global-univariate}, has the benefit of overcoming many of the hindrances of a multivariate model.
It does not care which product category you feed it, so it is easily scalable. \cite{Bandara2017}
states that a single global model might be detrimental to overall accuracy, but could be overcome by clustering product categories together.
%The literature is inconsistent in defining characteristics of a domain that would benefit from a global forecasting model.
%No empirical results have been found on the topic.
\cite{Rabanser2020}
shows that global models will improve accuracy, even if the global model assumption is not satisfied.

If grouping time-series together in clusters, another question is how this similarity function should look like.
\cite{Bandara2017} suggests a K-means clustering technique
which looks at several time-series characteristics, such as
strength of a trend, the strength of spikiness, strength of curvature,
and sales quantile.


The global method will not capture interdependent relationships directly
but might do so indirectly. It also has the potential to build a lot more complicated model,
as the amount of training data will drastically increase, which again will make the model less prone to
overfitting.

% Should this be under multi vs univariate section?
% Currently it does not have any references
One last approach is to build a multivariate model which only forecasts a single product category at the time.
The input to such a model might be a decomposition of the target time-series.
For example, decomposing a time-series into a component that shows how the series's trend behaves over time,
one component for the cycles, and one for noise.
Another option is to rely on external data sources which correlate with the target series.
For example, it is to be expected that if a product category group suddenly spikes in interest at Prisguiden.no,
then a similar spike would happen at Google.
Or bad weather might affect interest rates at Prisguiden because people are inside on the computer more.
Such a method can also have the additional benefit of being a global model.
This is precisely what \cite{Laptev} did, and they achieved a promising result.

To conclude,
the literature is inconsistent in defining characteristics of a domain that would benefit from a global forecasting model.
No empirical results have been found on the topic. Experiments could answer what model structure would best fit our domain.


