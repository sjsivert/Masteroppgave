
\subsection{Convolutional Autoencoder with LSTM}
\label{section:Discussion:Discussion:CNN-AE-LSTM}

% TODO
% - Why was the model selected?
% - Regarding the different datasets -> Does it make sense to use this model? (Is there high noise)
% - What are the results for the different data-sets (dataset 1, 2 and 3)
% - What model should performe better than others, and why?
% --> Do we have background theory to back this up?
% --> Does it make sense?
% --> 


% What was the aim of the experiments?
% --> Apply the CNN-AE and LSTM model to a new dataset with real world applicaitons
% --> Verrify the ability of ther hybrid model against the baseline LSTM
% --> Thus verrifying the claims done by "Zhao2019"
Inspired by the work done by \cite{Zhao2019}, the hybrid model CNN-AE and LSTM was selected.
The paper we base this work on explore the ability of the hybrid model to make time series predictions
on data with a high noise.
This is due to the first part of the model, the convolutional autoencoder.
The autoencoder is added in an attempt to reduce the noise contained in the data before the data is passed through to the LSTM model.
With the results from the paper by \cite{Zhao2019} claim the hybrid model vastly improves the predictions made, reducing the loss value with
several percentage points over a standard LSTM model, the hope was that this could be werrified on the real world practical application of
predicting trend at ``Prisguiden.no''.
The experiments done with the autoencoder is therefor compared to the baseline LSTM model in order to explore and verrify the claims done by
\cite{Zhao2019}.

% Link this to RQ-5 (CNN-AE and LSTM against the baseline LSTM)
% The paper is based on "Local univariate" models. We therefor first mention this
% All withing the margin of error
% We are not able to recreate the values by Zhao2019

% ....
% Dataset   -   LSTM   -  CNN
% Dataset 1 -  0.239   -  0.208 ->  13.9%
% Dataset 2 -  0.633   -  0.716 ->  -12.3%
% Dataset 3 -  0.488   -  0.483 ->  1%
After doing experiments with the LSTM and the CNN-AE and LSTM model on the three datasets used in this paper,
the results from the predictions is pressented in \cref{section:Results}.
Using the hybrid model, it is clear that we are not able to recreate the values pressemted in \cite{Zhao2019}
with all the different datasets.
The results from the experiments conducted on the different datasets wary vastly between the hybrid model and the LSTM model.
While the convolutional autoencoder and LSTM proves to have a bit more dificulties on dataset 2, and negligible improvements on dataset 3,
it does on average result in vast improvements on predictions done on dataset 1.
Using the hybrid model on dataset 1, the loss metrics for predictions are on average reduced with 13.9\% using the sMAPE metric.
In contrast, predictions made on dataset 2 result in predictions that on average are 12.3\% worse than the standard LSTM.
Lastly, predictions made on dataset 3 only improves with 1\%.

% Why is this?
% --> Why is it only better on dataset 1, and much worse on dataset 2
% --> Can the dataset have something to do with this?
% --> What is with the dataset from the Zhao2019 paper? Is this a lot easier? Is there more unneeded noise data?
The reasons behind behind the varying predictive ability of the convolutional autoencoder and LSTM lies partially
in the properties of the different datasets.
Although datasets 1 and 2 were selcted due to the autocorrelation between the time-series within the datasets,
they also have other abilities that differ the two.
\todo[inline]{Why is the model so much beter on dataset 1, and so much worse on dataset 2?}




  



% We have now seen how it works on univariate local models
% We use new models, exploring multivariate and global
% It is not always better in these cases
% How do we explain this?
% Can we compare directly with the work done in Zhao2019? What error metrics were used in these models?
While the paper \cite{Zhao2019} only focus on applying local univariate models,
we wished to explore the expansion of the use of the hybird model with other configurations.
These inclde the use of multivariate models and global models for each dataset.
Exploring the use of such variations to the Convolutional autoencoder and LSTM,
we hoped to uncover cases where the model performance is increased over the standard local uniariate model.



\subsubsection{Global univariate models}
% First -> Use of global models
% -> How does the global models performe in comparison to the LSTM version
% -> How does the global models performe in comparison to the local univariate standard model?
The first model variation that was tested were the use of a gloal univariate model.
Both the LSTM model and the hybird convolutional autoencoder and LSTM model were tested on the same datasets,
and with the same global univariate configuration.
This is done in order to be able to compare the global univariate hybrid model both to the local univariate model,
but also with the correlating global univariate LSTM model.

% ....
% Dataset   -   LSTM   -  CNN
% Dataset 1 -  0.203   -  0.210   ->  -3.4 %
% Dataset 2 -  0.662   -  0.675   ->  -1.9%
% Dataset 3 -  0.464   -  0.417   ->  10.7 %
Unlike the local univariate models described above, the hybrid model performe quite similar, although somewhat worse than the LSTM model
on both dataset 1 and 2.
However, dataset 3 offers an improvement of a little over 10\% of the sMAPE error metric by the hybrid model compared to the LSTM.

\todo[inline]{Why? Does it make sense that we get these results here?}

\todo[inline]{How does the model compare with the basic local univariate hybird model?}





\subsubsection{Local multivariate models}
\dots
% Second -> Use of local multivariate
% ....
% Dataset   -   LSTM   -  CNN
% Dataset 1 -  0.183   -  0.181  ->  1.1%
% Dataset 2 -  0.603   -  0.743  ->  -20.8%
% Dataset 3 -  0.325   -  0.354  ->  -8.5%


\todo[inline]{Why? Does it make sense that we get these results here?}

\todo[inline]{How does the model compare with the basic local univariate hybird model?}







\subsubsection{Global Multivariate models}
\dots
% Third -> Use of global multivariate
% ....
% Dataset   -   LSTM   -  CNN
% Dataset 1 -  0.202   -  0.198  ->  2%
% Dataset 2 -  0.642   -  0.703  ->  -9.1%
% Dataset 3 -  0.447   -  0.566  ->  -23.5%



\todo[inline]{Why? Does it make sense that we get these results here?}

\todo[inline]{How does the model compare with the basic local univariate hybird model?}







% Additional  ->  High variance vs low-variance datasets



% Old
\iffalse
The CNN-AE-LSTM model is only the best performing model for one dataset, dataset 1.
Here the CNN-AE-LSTM local multivariate model structure beat the plain local multivariate LSTM with $1.09\%$.
The CNN-AE-LSTM performs competitively on all the experiments on dataset 1 except for the global univariate model structure.
Our experiments show that adding a Convolutional Autoencoder to an LSTM can be beneficial in some scenarios.


\begin{figure}[h!]
  \centering
  \includegraphics[width=\textwidth]{./figs/code_generated/data_exploration/cat-id-10320-plot.png}
  \hfill
  \caption{TODO}
  \label{fig:cat-id-10320-cnn-ae-beat-lstm}
\end{figure}

\import{./tables/results/dataset_high_variance}{Average-metric-dataset-high-variance.tex}

\begin{figure}[h!]
  \centering
  \includegraphics[width=\textwidth]{./figs/results/boxplot/smape-dataset_high_variance.png}
  \hfill
  \caption{TODO}
  \label{fig:results-smape-dataset-high-variance}
\end{figure}
\fi
