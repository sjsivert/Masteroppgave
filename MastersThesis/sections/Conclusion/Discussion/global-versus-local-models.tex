

\subsection{Global versus Local models}
Comparing the local univariate against the global univariate model across all datasets,
the global model has a sMAPE performance increase of $2.4\%$ on dataset 1, $3.2\%$ on dataset 2 and
$4.9\%$ on dataset 3. This is surprising as the performance increase does not seem to correlate
with the how homogeneous the time series in the dataset are to each other.
Dataset 1, which is the most homogeneous set of them all has the least performance increase.
What might explain these results is that the performance increase is closely connected to the amount
of data available. Dataset 1 consists of the longest time series.
Dataset 3 has the least amount of data
\todo[inline]{Fact check these claims, and add the time series lengths to back up!}
These results support \cite[]{Montero-Manso2021} preposition that global models can give
improve forecasting accuracy, even if the strong assumption that the time series
is generated by the same process.

The same performance increase is not to be found on the multivariate models.
The local multivariate models are really good at capturing the trend and seasonality of
a given product category. This does not seem to translate well to a global model.
One might expect that global models should be able to learn seasonality across
time series if the series contains the same seasonality. But our results show the opposite.
The dataset that suffer the most from making a multivariate local model global is dataset 3
with a sMAPE perfomance loss of $-37\%$. Dataset 1 and dataset 2 got $-10.38\%$ and $-6.47\%$ loss
respectively.
One explanation for this might be that even though all categories in dataset 3 are popular during the
winter, and peaks around the same months, their seasonality is not enough in sync.
For example \textit{"Vintersko"} and \textit{"Vinterjakke"} has their biggest peaks around oktober
when the weather starts becoming cold. While \textit{"Langrennski"} and \textit{"Skisko"}
peaks around january, when the snow starts falling. For a NN which can not differentiate between
which category it is looking at, this will seem like conflicting information which will
hurt the NNs modeling capability.



TODO:
% Write about which dataset the global model performs best on.
%The global models seems to be doing be doing...
%
%Reasons for why global models does not perform better?
%The time series consist of enough data for the local models to generalize
%\cite{Montero-Manso2021}.

% Difference between MASE ans sMAPE
The global models seem to have a better 1 day MASE, but a lower sMAPE.
This is likely because sMAPE is not a symetric metric, as it punishes
under forcasts higher than over forecasts. Looking at the predictions made by
the local and global models, it seems that the global models in general tends to
under-forcast, and the local models has a tendency to over-forecast.

\subsection{LSTM trend and seasonality}
\begin{itemize}
  \item LSTM and ARIAM seem to have trouble with datasets with yearly seasonalities
  \item {LSTM will perform significantly better on these datasets if additional
        a multivariate version is used with additional information about date}
  \item {If date is not available then detrending the dataset using differencing is a good second alternative}
\end{itemize}

