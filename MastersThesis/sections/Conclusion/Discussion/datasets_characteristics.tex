
\subsection{Datasets Characteristics}
The results we get reveals a lot of information about the Characteristics of the different datasets.
The sMAPE is our best indicator for how well the prediction graph fits the target graph.
The 1-day MASE gives us an indicator for how well our predictions are compared to the naive
prediction. It is worth noting that how good the naive prediction is differs from time series
to time series. In a random-walk the naive prediction will be the best possible prediction.
This is also true for time series with a high autocorrelation coefficient.
The same is true for the 7-day MASE. A time series with a strong weekly seasonality will
give a higher MASE compared to a time series with a low weekly seasonality.
Therefore comparing MASEs from different datasets can be misleading without knowledge of
the underlying characteristics of these series.

Looking at the results we can make some educated guesses of these time series characteristics.
We get the best forecasts fits from dataset 1, which has a mean sMAPE of $0.2034$.
Second comes dataset 3 with a sMAPE of $0.4362$. The hardest dataset to forecast was dataset 2 with a
sMAPE of $0.697$. Looking at the 1-day MASE we can expect the mean autocorrelation coefficient for
dataset 3 to be the highest among the three, with a MASE of $2.009$.
Calculating the autocorrelation for all the time series in the datasets and
taking the average of the results confirms our hypothesis. The results are shown in \Cref{tab:datasets-autocorrelation}.
Dataset 3 has the highest autocorrelation of all the datasets with a mean
of $0.93$. There is also a relatively strong correlation between the 1-day MASE results
from the local univariate LSTM method and the time series autocorrelation.

\begin{table}[htbp]
  \begin{center}
    \begin{tabular}{|c|c|c|}\hline\hline
      Dataset & Mean Autocorrelation & Autocorrelation correlation with MASE \\\hline
      1       & 0.753                & 0.229                                 \\\hline
      2       & 0.775                & 0.295                                 \\\hline
      3       & 0.930                & 0.304                                 \\\hline
    \end{tabular}
    \caption{Mean autocorrelation and the correlation of the MASE results from Local Univariate LSTM and the autocorrelation.}
  \end{center}
  \label{tab:datasets-autocorrelation}
\end{table}%

\todo[inline]{Fetch autocorrelation for the different datasets and check our guesses.}

In general, using sMAPE as metric, dataset 2 seems to be the most difficult dataset to forecast.
This makes sense when because the dataset consists of many vastly different time series
with few apparent patterns.