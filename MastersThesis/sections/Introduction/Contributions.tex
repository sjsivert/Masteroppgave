\section{Contributions}
\label{sections:Introduction:Contributions}


% What is the aim of the thesis? What is it intended to solve?
% Contribute to state of the art
% Propose new method for e-commerce, and test ot on a dataset
% Assess the new method agains current state of the art methods found with literature search
% We evaluate current state of the art models, then use these to compare against our own model

The main focus of this thesis is to assess current state-of-the-art time series prediction in E-commerce forecasting,
and make a comparison with a new state-of-the-art predictive method.
Contributing to state of the art,
we propose the use of a new predictive method in E-commerce forecasting.
Introducin a predictive method that has yet to be applied in an E-commerce setting,
we evaluate this new method against proven predictive methods that is currently considered state-of-the-art.

The main contributions of this thesis is:
\begin{enumerate}
  \item {\it To To evaluate and compare current methods of time-series prediction on E-commerce forecasting.}
  \item {\it Formulate a framework for achieving higher predictive accuracy than the current state-of-the-art methods on our problem space.}
  \item {\it Compare current state-of-the-art methods against a new predictive model to evaluate predictive ability.}
\end{enumerate}




\iffalse
The main focus of this work is to assess the current state of time-series prediction in E-commerce forecasting.
Contributing to the current state of the art,
we propose a method for increasing the predictive ability of time-series forecasting
assessing interest trends of e-commerce product categories.
The main contributions of this paper are:

\begin{enumerate}
  \item {\it To evaluate and compare current methods of time-series prediction on e-commerce.}
  \item {\it To evaluate and compare current state-of-the-art methods for time-series forecasting.}
  \item {\it Formulates a framework for achieving higher predictive accuracy than the current state-of-the-art methods on our problem space.}
\end{enumerate}
\fi


%%%%%%%%%%%%%%%%%%%%%%%%%%%%%%%%%%%%%%%%%%%%%%%%%%%%%%%%%%%%%%%%%%%%%%%%%%%%%%%%%
\iffalse
  The main description of the contributions will come in \Cref{cont} after the results are presented. This section just provides a brief summary of the main contributions of the work. This section can also be left out, leaving all discussions in \Cref{cont}.

  The format of this section will generally follow the following format:
  {\it
  Donec non turpis nec neque egestas faucibus nec id neque. Etiam consectetur, odio vitae gravida tempus, diam velit sagittis turpis, a molestie ligula tellus at nunc. Nam convallis consequat vestibulum. Proin dolor neque, dapibus a pellentesque a, commodo a nibh.}

  \begin{enumerate}
    \item {\it Lorem ipsum dolor sit amet, consectetur adipiscing elit.}
    \item {\it Lorem ipsum dolor sit amet, consectetur adipiscing elit.}
    \item {\it Lorem ipsum dolor sit amet, consectetur adipiscing elit.}
  \end{enumerate}
\fi
