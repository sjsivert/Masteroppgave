\section{Data Exploration*}
\label{section:Data:DataExploration}
In this section we will describe the main dataset given to us
by Prisguiden.no. The full Jupyter notebook exploration
is located at \cite{notebook-data-exploration}. This section will only
describe the highlights from that Notebook.

\subsection{Basic data structure}
As of \textit{26-04-2022} the dataset consists of 41941504 entries.
Most of the categories consist of roughly $1239$ data points, depending on when the category
was created.
The oldest recorded data are from the beginning of 2019.
It consists of 9 features.
as shown in \Cref{table:features_market_insight_dataset}.
\begin{table}[H]
  \centering
  \caption{Features of the Market insight dataset}
  \label{table:features_market_insight_dataset}
  \begin{tabular}{|c|l|c|l|}\hline
    Nr & Name            & Type             & Description                                 \\ \hline
    1  & id              & (int64)          & Unique identifier                           \\ \hline
    2  & product id      & (int64)          & Associated Product id                       \\ \hline
    3  & manufacturer id & (int64)          & Manufacturer id                             \\ \hline
    4  & cat id          & (int64)          & Associated category id                      \\ \hline
    5  & root cat id     & (int64)          & Associated root category id                 \\ \hline
    6  & date            & (datetime64[ns]) & The date the data was captured              \\ \hline
    7  & hits            & (int64)          & How many times the product page was visited \\ \hline
    8  & clicks          & (int64)          & How many times users clicked to a retailer  \\ \hline
    9  & last modified   & (object)         & When the row was last modified              \\ \hline
  \end{tabular}
\end{table}

It is worth noticing that hits and clicks are different features that measure the same thing: user interest.
A hit is how many times the product page on Prisguiden.no is visited.
A click is how many times a user follows a link from Prisguiden.no to an external retailer for a given product.
A product can receive a hit and not a click if the user just visits a product detail page without clicking to a retailer.
A product may receive a click without a hit if the user clicks on an AD or a campaign, which will lead the user directly to the retailer,
skipping the product page on Prisguiden.no.
The correlation between hits and clicks for all categories is roughly $0.578$.

The dataset consists of 1325 unique categories and 310499 unique products.
Each product is associated with a product category; for example,
all CPUs are associated with the category
\textit{"Prosessor (CPU)"s}.
Each product category is a leaf node of a category hierarchy.
The category \textit{"Prosessor (CPU)"} is a child of the parent category
\textit{"Datakomponenter"}, which itself is a child of \textit{"Data"}.

% Prisguiden.no are not interested in insight on the product level, but on the product category level.
Summing together all product clicks and hits to its closest parent category
gives us a table on the format shown in \Cref{table:market_insights_overview_11-12-21}.

\import{./tables/code_generated/data_exploration/}{market_insights_overview_11-12-21.tex}

\subsection{Category plot analysis}
\Cref{fig:lineplot1} % and \Cref{fig:lineplot2}
show hits and clicks
from a random sample of categories from 2019 to 2021.
Most of the plots show a clear yearly periodic pattern, as shown in
\Cref{fig:lineplot-Hodetelefoner},
\Cref{fig:lineplot-Mobiltelefon},
\Cref{fig:lineplot-Julekalender}.
The scale of values differs between categories.
% While \textit{"Varmepumpe"} \Cref{fig:lineplot-Varmepumpe} hower around 1000 hits per day,
The category \textit{"Mobiltelefon"} \Cref{fig:lineplot-Mobiltelefon} gets around 7500 hits per day.
Meanwhile, the values within each category can differ vastly.
\textit{"Julekalender"} \Cref{fig:lineplot-Julekalender} has a hit peak around 16000,
while it usually gets around 0 hits per day.

Some categories experience extreme variations in traffic at different intervals.
One example of this is the category \textit{"Grafikkort GPU"}.
The traffic tracked before and after Christmas 2020 differs vastly,
and the data changes its behavior and distribution from then on.

\Cref{table:Mobiltelefon_statistics} shows some basic statistics for \textit{"Mobiltelefoner"}.
This exact category was chosen at random just to get an idea of how one-time-series might behave.
The time-series has a mean of 6453 hits and a standard deviation of 1881, which is
around 30\% of the mean. This is quite a big variance in the dataset.

% Mobiltelefon stats table
\import{./tables/code_generated/data_exploration/}{Mobiltelefon_statistics.tex}

It is interesting to see how many categories receive zero values and undefined values each day.
A zero value count in this context means how many days a given category has gotten 0 hits, but it has gotten some clicks.
A NaN value in this context means that the given category has gotten 0 clicks and 0 hits on the given day.
\Cref{fig:category_0_and_NaN_values} shows a lineplot of the different distributions.
In general, relatively few categories have 0-values.
Some of the first categories created, the ones with id 0-500, there exists some days with 0 interest.
However, most categories have at least one click and one hit each day.
Categories with id above 10000 start to see a lot more zero values.
The most likely explanation for this is that these categories are created at a later point in time.
The steadily growing line at the end of \Cref{fig:category_NaN_values} near id 12000 supports this theory.


\begin{figure}[H]
  \centering
  \caption{Counting how many categories have days with 0 hits or NaN values}
  \label{fig:category_0_and_NaN_values}
  \begin{subfigure}[b]{0.47\textwidth}
    \includegraphics[width=\textwidth]{./figs/code_generated/data_exploration/category_0_values.png}
    \hfill
    \caption{0-values}
    \label{fig:category_0_values}
  \end{subfigure}
  \begin{subfigure}[b]{0.47\textwidth}
    \includegraphics[width=\textwidth]{./figs/code_generated/data_exploration/category_NaN_values.png}
    \hfill
    \caption{NaN-values}
    \label{fig:category_NaN_values}
  \end{subfigure}
\end{figure}



% Time series visualization
% I think we only need one of these
\begin{figure}[h!]
  \centering
  \caption{Category plots of hits and click rate from 2019-2021}
  \label{fig:lineplot1}
  \begin{subfigure}[b]{\textwidth}
    \includegraphics[width=\textwidth]{./figs/code_generated/data_exploration/lineplot_51_Hodetelefoner og ørepropper.png}
    \hfill
    \caption{Hits and clicks rate for \textit{"Hodetelefoner og ørepropper"}}
    \label{fig:lineplot-Hodetelefoner}
  \end{subfigure}

  \begin{subfigure}[b]{\textwidth}
    \includegraphics[width=\textwidth]{./figs/code_generated/data_exploration/lineplot_19_Mobiltelefon.png}
    \hfill
    \caption{Hits and clicks rate for \textit{"Mobiltelefon"}}
    \label{fig:lineplot-Mobiltelefon}
  \end{subfigure}

  \begin{subfigure}[b]{\textwidth}
    \includegraphics[width=\textwidth]{./figs/code_generated/data_exploration/lineplot_11781_Julekalender og adventskalender.png}
    \hfill
    \caption{Hits and clicks rate for \textit{"Julekalender"}}
    \label{fig:lineplot-Julekalender}
  \end{subfigure}
\end{figure}


\iffalse
  \begin{figure}[H]
    \centering
    \caption{Category plots of hits and click rate from 2019-2021}
    \label{fig:lineplot2}

    \begin{subfigure}[b]{\textwidth}
      \includegraphics[width=\textwidth]{./figs/code_generated/data_exploration/lineplot_43_Spillkonsoller.png}
      \hfill
      \caption{Hits and clicks rate for \textit{"Spillkonsoller"}}
      \label{fig:lineplot-Spillkonsoller}
    \end{subfigure}
    \begin{subfigure}[b]{\textwidth}
      \includegraphics[width=\textwidth]{./figs/code_generated/data_exploration/lineplot_30_gpu.png}
      \hfill
      \caption{Hits and clicks rate for \textit{"Grafikkort GPU"}}
      \label{fig:lineplot-GPU}
    \end{subfigure}

    \begin{subfigure}[b]{\textwidth}
      \includegraphics[width=\textwidth]{./figs/code_generated/data_exploration/lineplot_11054_Varmepumpe.png}
      \hfill
      \caption{Hits and clicks rate for \textit{"Varmepumpe"}}
      \label{fig:lineplot-Varmepumpe}
    \end{subfigure}
  \end{figure}
\fi


\subsection{Correlation among categories}
We did a correlation analysis to get an idea of how the categories correlate.
We can create a correlation matrix by aggregating the dataset to a pivot table, using dates as the row index and the products id's as columns.
To get reliable results, we limited the matrix to categories with more than 100 common data points.
Because of the large number of categories, we created two matrixes. The small matrix in \autoref{fig:category_corelation_matrix_10},
is made of a random subset of categories.
This small one should be easy to read and understand.
We also made a matrix of all the categories in \autoref{fig:category_corelation_matrix_full}. This matrix is too complex to read details from but gives a full picture of the data.
The lighter cells of the matrix indicate a correlation toward 1.0. It is worth noticing that the two matrixes do not use the exact same color spectrum.
A completely white tile indicates not enough data to calculate a correlation.

Looking at the smaller correlation matrix in \autoref{fig:category_corelation_matrix_10} we can see that
\textit{"Hylle"} correlates strongly  with \textit{"Bord"} with a corraltion of 0.7.
\textit{"Ryddesag"} and \textit{"Kontrollenhet og gateway"} does not correlate at all, with a correlation of 0.04.
\textit{"Singlet til barn"} does not have enough data yet, so the whole row is white, indicating missing values.

%Analyzing all categories with more than 100 common datapoints, the values are compared and used to create the correlation matrix.
%The matrix is shown in \autoref{fig:category_corelation_matrix}.
%The lighter areas of the matrix indicate a correlation towards 1.0, while the darker areas indicate a correlation towards -1.0.

Based on the matrix's color spectrum of the full correlation matrix in \autoref{fig:category_corelation_matrix_full}, it is clear that categories cover almost the whole spectrum of correlation relationships.
The overall bright colors indicate a bias towards positive correlation.

Also worth noticing from the big correlation matrix is an indication of more purple lines and areas to the lower right corner of the matrix.
Tech-related categories dominated Prisguiden's product category in its early years. In recent years they have expanded their reach to include more common household
products. It makes sense that these new products do not correlate much with technology products.

\begin{figure}[H]
  \centering
  \includegraphics[width=0.8\textwidth]{./figs/code_generated/data_exploration/category_correlation_matrix_10_categories.png}
  \hfill
  \caption{Correlation matrix of 10 random categories}
  \label{fig:category_corelation_matrix_10}
\end{figure}


\begin{figure}[H]
  \centering
  \includegraphics[width=0.8\textwidth]{./figs/code_generated/data_exploration/category_correlation_matrix_all_categories.png}
  \hfill
  \caption{Full category correlation matrix}
  \label{fig:category_corelation_matrix_full}
\end{figure}

